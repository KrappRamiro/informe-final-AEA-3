\documentclass[../main.tex]{subfiles}
\graphicspath{../imagenes}

\begin{document}
\renewcommand{\subsectionbreak}{}
\section{Código G}

	\subsection{Introducción}

	Código G es un lenguaje de programación utilizado para máquinas de control numérico
	por computadora (CNC).

	Se utiliza principalmente en la automatización debido a que los códigos G se les
	ordena acciones específicas para la máquina, como, por ejemplo, movimientos simples
	de la máquina que lo desplacen a otro punto del eje X, Y, o funciones de taladro, 
	empezar a ``comer'' el material, girar en cierto sentido, etc.

	\subsection{Grupos}

	Cada código G tienen un numero de grupo, estos grupos de código contiene comandos
	para un objetivo específico. Ejemplo, el grupo 2 ordena un movimiento a cierto punto
	de los ejes de la máquina X26 Y40.

	Cada uno de los grupos tiene un código G dominante, 
	referido como el código G predeterminado.

	Un código G predeterminado son los que la maquina utiliza en cada grupo, salvo que se 
	especifique otro código G del grupo. Por ejemplo, la programación de un movimiento
	X y Z,  X-2. Z-4 posicionara la maquina utilizando G00.

	\subsection{Instrucciones basicas}
	Las instrucciones basicas que utilizamos son: 
	\begin{itemize}
		\item T(N) Numero de herramienta que vamos a usar.
		\item M2 Finaliza el programa (necesita de M5).
		\item M3 Arranque del huesillo en sentido horario.
		\item M4 Arranque del huesillo en sentido anti-horario.
		\item M5 Detiene el huesillo.
		\item s(n) Velocidad de giro del huesillo (usamos s1).
		\item F(N) Velocidad del desplazamiento de mm/min (F100).
		\item G0 Velocidad de desplazamiento rapida o aerea, es usado para ir de un punto al otro.
		\item G1 Velocidad del mecanizado. es suficientemente lento para que pueda despejar
			el material sin flexionarla.
		\item G2 Movimiento circular en sentido horario.(Necesita la posicion X,Y y el radio de giro).
			La posicion inicial y final no puede ser la misma.
		\item G3 Movimiento circular, pero en sentido anti-horario.
		\item G5 suspende momentaneamente el programa.
		\item G90 suspende momentaneamente el programa.
		\item G91 Especifica que todo lo que ve escribe a continuación sera con el 
			sistemas de coordenadas respecto al origen(0,0).
		\item G92 Cambia de cero matematico por programacion, permite redefinir la posicion actual,
			es util para copiar y pegar un bloque de codigo.
	\end{itemize}

	\clearpage
	
	\subsection{Codigo de ejemplo}
\inputminted
[frame= lines, linenos, breaklines, tabsize = 3, fontsize=\footnotesize, 
label=Código de ejemplo, fontseries=inconsolata]
{gcode}{codigo_g.gcode}


\end{document}
